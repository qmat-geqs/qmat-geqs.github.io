%%%%%%%%%%%%%%%%%%%%%%%%%%%%%%%%%%%%%%%%%%%%%
%       GENERAL TEMPLATE SETTINGS           %
%%%%%%%%%%%%%%%%%%%%%%%%%%%%%%%%%%%%%%%%%%%%%

% Do not change the settings!

\documentclass[
 reprint,
 unsortedaddress,
 preprintnumbers,
 nobibnotes,
 nofootinbib,
 longbibliography,
 amsmath,amssymb,
 aps,pra,
 onecolumn
]{revtex4-2}

\usepackage[french]{babel}
\usepackage[T1]{fontenc}

\renewcommand{\journalname}{Strasbourg Students Physical Letters}
\renewcommand{\volumename}{X}
\renewcommand{\numbername}{04xx}
\newcommand{\header}{{\footnotesize\today} \hspace{7em} {\sc\journalname} \; {\bf\volumename},\; \numbername \; \hspace{9em} {\footnotesize CC BY-SA 4.0}}

% Definitions related to bibliography format
\def\sspl{Stras.\ Stud.\ Phys.\ Lett.}%

% amsmath and amssymb are charged by the document class
\usepackage{physics}% Dirac notation, abreviations for derivatives, parentheses, etc.
\usepackage{siunitx}% Normalised notations for physical units
\usepackage{bm}% Bold mathematical characters like \bm\alpha
\usepackage{chemfig}% Molecules in different representations, reaction mechanisms, etc.
\usepackage{tikz-cd}% Algebraic diagrams
\usepackage{graphicx,subcaption}% Include figure files
\usepackage{dcolumn}% Align table columns on decimal point
\usepackage{hyperref}% For hyperlinks

%%%%%%%%%%%%%%%%%%%%%%%%%%%%%%%%%%%%%%%%%%%%%
%     END OF GENERAL TEMPLATE SETTINGS      %
%%%%%%%%%%%%%%%%%%%%%%%%%%%%%%%%%%%%%%%%%%%%%

%%%% PERSONAL COMMANDS AND PACKAGES %%%%
%\usepackage{mypackage}
%\input{mypreamble}
%\newcommand{\mycommand}{}


\begin{document}

\preprint{\header}% Do not change this setting

%%%%%%%%%%%%%%%%%%%%%%%%%%%%%%%%%%%%%%%%%%%%%
%                FRONTMATTER                %
%%%%%%%%%%%%%%%%%%%%%%%%%%%%%%%%%%%%%%%%%%%%%

% Replace the fields with your title, name, etc. and remove the fields that you don't need

\title{Modèle pour les notes de cours des 2SPL}
%\thanks{}% Facultative

% The authors are the persons who compiled and revised the notes.

\author{Premier Auteur}
\email{email1@institution.org}% Facultative
\affiliation{Faculté de Physique et Ingénierie, Université de Strasbourg}

\author{Révisé et augmenté par Second Auteur}% Facultative
\email{email2@institution.org}% Facultative
\affiliation{Faculté de Physique et Ingénierie, Université de Strasbourg}

\begin{abstract}
    Notes inspirées par le cours de [Pr. Super Professeur] sur la [mécanique quantique (L3S5)] en [année]. [Plus d'informations sur le contenu des notes peuvent être fournies dans l'abstract.]\\
    En soumettant ces notes de cours aux {\em Strasbourg Students Physical Letters}, les auteurs certifient qu'ils ont obtenu l'accord de l'enseignant, et acceptent eux-mêmes qu'elles soit distribuées sous la licence CC-BY-SA 4.0 (Distribution libre - Attribution - Partage sous les mêmes conditions) sur le site des 2SPL (\url{www.2spl.fr}).\\
    Ces notes n'ont pas été corrigées par l'enseignant et peuvent contenir des erreurs ou dévier du contenu du cours. L'enseignant ne peut être tenu pour responsable pour d'éventuelles erreurs ou approximations contenues dans ces notes.
\end{abstract}

\maketitle

%%%%%%%%%%%%%%%%%%%%%%%%%%%%%%%%%%%%%%%%%%%%%
%             END OF FRONTMATTER            %
%%%%%%%%%%%%%%%%%%%%%%%%%%%%%%%%%%%%%%%%%%%%%

% \begin{acknowledgments}
%     Facultative acknowledgments section.
% \end{acknowledgments}

\tableofcontents

% The bibliography should be laid out in the following format, we do not accept .bib files for lecture notes.

\section*{Bibliographie}

\begin{itemize}
    \item J. Polonyi, \textit{Quantum Mechanics II}, lecture notes, \url{https://wigner.hu/~polonyi.janos/qm2.pdf}.
    \item L. D. Landau, E. Lifchitz, \textit{Cours de physique théorique}, 2nd ed., Vol. III~: Mécanique quantique (Mir, Moscou, 1967).
    \item J. A. Wheeler, R. P. Feynman, Classical electrodynamics in terms of direct interparticle action, \rmp, {\bf 21}, 425 (1949), doi:10.1103/RevModPhys.21.425.
    \item S. Beaudoin, Extension of the standard model with the addition of the Weinberg operator, \sspl, {\bf 1}, 0101 (2024).
\end{itemize}

\section*{Introduction}

L'introduction doit donner le contexte scientifique du cours. On pourra employer des notes de bas de page pour donner des précisions~\footnote{Lorem ipsum}, tandis que les annexes (comme l'annexe~\ref{app:lorem_appendix}) peuvent être utilisées pour présenter des développements qui n'étaient typiquement pas présents dans le cours d'origine.

\section{Première partie}

Les équations doivent être numérotées:
\begin{equation}
    m \dv{\bm\gamma}{t} = \mathbf{F} ,
\end{equation}
éventuellement collectivement:
\begin{equation}
\begin{split}
    \delta \expval{\hat{H}}{\psi} &= \mel{\delta \psi}{\hat{H}}{\psi} + \mel{\psi}{\hat{H}}{\delta \psi} \\
    &= 2 \Re(\mel{\delta \psi}{\hat{H}}{\psi})
\end{split}
\end{equation}
ou par des sous-équations:
\begin{subequations}
\begin{gather}
    \partial_\mu F^{\mu\nu} = J^\nu, \\
    F^{\mu\nu} = g^{\mu\sigma} g^{\nu\tau} \pqty{ \partial_\sigma A_\tau - \partial_\tau A_\sigma }.
\end{gather}
\end{subequations}

\section{Deuxième partie}

Les images sont attendues en format PDF, ou éventuellement en format PNG ou JPG.

% \begin{figure}
%     \centering
%     \includegraphics[width=\columnwidth]{figure.pdf}
%     \caption{Caption}
%     \label{fig:placeholder}
% \end{figure}

Des tables comme la table~\ref{tab:gates} peuvent être utilisées pour présenter des informations.

\begin{table}[ht]
\begin{ruledtabular}
    \centering
    \begin{tabular}{ccc}
        Symbol & Definition & Name \\ \hline
        \multicolumn{3}{l}{\bf Oscillator-only gates} \\ \hline
        $\hat{D} (\xi)$ & $\exp( \xi  b^\dag - \xi^\ast b )$ & Displacement \\
        $\hat{R} (\theta)$ & $\exp( {\rm i} \theta  b^\dag b )$ & Rotation\footnote{$\theta = \theta_\text{strawberry} = \theta_\text{qiskit} = -\theta_\text{Liu/Crane}$} \\
        $\hat{S} (\xi)$ & $\exp( \frac{1}{2} (\xi^\ast  b^2 - \xi b^{\dag2}) )$ & Squeezing \\
        $\widehat{\rm BS}_{\alpha,\beta} (\theta,\varphi)$ & $\exp( \theta ( {\rm e}^{{\rm i}\varphi} b_\alpha b_\beta^\dag - {\rm e}^{-{\rm i}\varphi} b_\alpha^\dag b_\beta) )$ & Beamsplitter\footnote{$\theta = \theta_\text{strawberry} = \abs{\xi_\text{qiskit}} = \theta_\text{Liu}/2 = \theta_\text{Crane}$}\footnote{$\varphi = \varphi_\text{strawberry} = - \arg{\xi_\text{qiskit}} + \pi = - \varphi_\text{Liu/Crane} + \pi/2$} \\ \hline
        \multicolumn{3}{l}{{\bf Oscillator-qubit gates}} \\ \hline
        $\widehat{\rm CD}_{\alpha,i} (\xi)$ & $\exp( (\xi  b_\alpha^\dag - \xi^\ast b_\alpha) Z_i )$ & CDisplacement
    \end{tabular}
    \caption{Some native gates in superconducting quantum hardware. $\xi\in\mathbb{C}$, $\theta\in [0;2\pi[$, $\varphi\in[0;\pi[$. $\alpha,\beta$ are qumode indices, $i,j$ are qubit indices.}
    \label{tab:gates}
\end{ruledtabular}
\end{table}


\appendix


\section{Première annexe}
\label{app:lorem_appendix}

Lorem ipsum dolor sit amet, consectetur adipiscing elit, sed do eiusmod tempor incididunt ut labore et dolore magna aliqua. Ut enim ad minim veniam, quis nostrud exercitation ullamco laboris nisi ut aliquip ex ea commodo consequat. Duis aute irure dolor in reprehenderit in voluptate velit esse cillum dolore eu fugiat nulla pariatur. Excepteur sint occaecat cupidatat non proident, sunt in culpa qui officia deserunt mollit anim id est laborum.

\end{document}
