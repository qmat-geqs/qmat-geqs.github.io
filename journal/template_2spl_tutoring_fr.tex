%%%%%%%%%%%%%%%%%%%%%%%%%%%%%%%%%%%%%%%%%%%%%
%       GENERAL TEMPLATE SETTINGS           %
%%%%%%%%%%%%%%%%%%%%%%%%%%%%%%%%%%%%%%%%%%%%%

% Do not change the settings!

\documentclass[
 reprint,
 preprintnumbers,
 nobibnotes,
 nofootinbib,
 longbibliography,
 amsmath,amssymb,
 aps,prl,
 onecolumn
]{revtex4-2}

\usepackage[french]{babel}
\usepackage[T1]{fontenc}

\renewcommand{\journalname}{Strasbourg Students Physical Letters}
\renewcommand{\volumename}{X}
\renewcommand{\numbername}{03xx}
\newcommand{\header}{{\footnotesize\today} \hspace{8em} {\sc\journalname} \; {\bf\volumename},\; \numbername \; \hspace{9em} {\footnotesize CC-BY-SA 4.0}}

% Definitions related to bibliography format
\def\sspl{Stras.\ Stud.\ Phys.\ Lett.}%

% amsmath and amssymb are charged by the document class
\usepackage{physics}% Dirac notation, abreviations for derivatives, parentheses, etc.
\usepackage{siunitx}% Normalised notations for physical units
\usepackage{bm}% Bold mathematical characters like \bm\alpha
\usepackage{chemfig}% Molecules in different representations, reaction mechanisms, etc.
\usepackage{tikz-cd}% Algebraic diagrams
\usepackage{graphicx,subcaption}% Include figure files
\usepackage{dcolumn}% Align table columns on decimal point

%%%%%%%%%%%%%%%%%%%%%%%%%%%%%%%%%%%%%%%%%%%%%
%     END OF GENERAL TEMPLATE SETTINGS      %
%%%%%%%%%%%%%%%%%%%%%%%%%%%%%%%%%%%%%%%%%%%%%

%%%% PERSONAL COMMANDS AND PACKAGES %%%%
%\usepackage{mypackage}
%\input{mypreamble}
%\newcommand{\mycommand}{}


\begin{document}

\preprint{\header}% Do not change this setting

%%%%%%%%%%%%%%%%%%%%%%%%%%%%%%%%%%%%%%%%%%%%%
%                FRONTMATTER                %
%%%%%%%%%%%%%%%%%%%%%%%%%%%%%%%%%%%%%%%%%%%%%

% Replace the fields with your title, name, etc. and remove the fields that you don't need

\title{Modèle pour les notes de tutorat des 2SPL}
\thanks{Organisé par le Pôle Tutorat de la Faculté de Physique et Ingénierie, Université de Strasbourg.}

\author{Premier Auteur}
%\email{email1@institution.org}% Facultative
\affiliation{Faculté de Physique et Ingénierie, Université de Strasbourg}

% \author{Second Auteur}
% \email{email2@institution.org}% Facultative
% \affiliation{Faculté de Physique et Ingénierie, Université de Strasbourg}

% Replace the parts in brackets with the appropriate information
\begin{abstract}
    Notes pour la séance de tutorat du [1er avril 2026], sur le cours de [L1 mathématiques, L3 mécanique quantique...]. Sujets abordés~: [multiplication de matrices].
\end{abstract}

\maketitle

%%%%%%%%%%%%%%%%%%%%%%%%%%%%%%%%%%%%%%%%%%%%%
%             END OF FRONTMATTER            %
%%%%%%%%%%%%%%%%%%%%%%%%%%%%%%%%%%%%%%%%%%%%%


\section{Titre de section}

Les notes de tutorat peuvent être structurées comme le(s) auteur(s) le souhaite(nt).

\section{Autre section}

Les images (s'il y en a) sont attendues au format PDF, ou éventuellement en PNG ou JPG.

% \begin{figure}
%     \centering
%     \includegraphics[width=\columnwidth]{figure.pdf}
%     \caption{Légende}
%     \label{fig:placeholder}
% \end{figure}

Des tables comme celles de la ref.~\ref{tab:gates} peuvent aussi être employées pour présenter des informations.

\begin{table}[ht]
\begin{ruledtabular}
    \centering
    \begin{tabular}{ccc}
        Symbole & Définition & Nom \\ \hline
        \multicolumn{3}{l}{\bf Portes qumode} \\ \hline
        $\hat{D} (\xi)$ & $\exp( \xi  b^\dag - \xi^\ast b )$ & Déplacement \\
        $\hat{R} (\theta)$ & $\exp( {\rm i} \theta  b^\dag b )$ & Rotation \\
        $\hat{S} (\xi)$ & $\exp( \frac{1}{2} (\xi^\ast  b^2 - \xi b^{\dag2}) )$ & Compression \\
        $\widehat{\rm BS}_{\alpha,\beta} (\theta,\varphi)$ & $\exp( \theta ( {\rm e}^{{\rm i}\varphi} b_\alpha b_\beta^\dag - {\rm e}^{-{\rm i}\varphi} b_\alpha^\dag b_\beta) )$ & Diviseur de faisceau \\ \hline
        \multicolumn{3}{l}{{\bf Portes qubit-qumode}} \\ \hline
        $\widehat{\rm CD}_{\alpha,i} (\xi)$ & $\exp( (\xi  b_\alpha^\dag - \xi^\ast b_\alpha) Z_i )$ & C. Déplacement
    \end{tabular}
    \caption{Quelques portes natives dans le {\em hardware} supraconducteur. $\xi\in\mathbb{C}$, $\theta\in [0;2\pi[$, $\varphi\in[0;\pi[$. $\alpha,\beta$ sont des indices de qumodes, $i,j$ sont des indices de qubits.}
    \label{tab:gates}
\end{ruledtabular}
\end{table}



% Facultative acknowledgment section
% \begin{acknowledgments}
%     Je souhaiterais remercier...
% \end{acknowledgments}

\section{Licence}

En soumettant cet article aux {\em Strasbourg Students Physical Letters}, les auteurs acceptent qu'il soit distribué sous la licence CC-BY-SA 4.0 (Distribution libre - Attribution - Partage sous les mêmes conditions) sur le site des 2SPL (\url{www.2spl.fr}).


\end{document}
